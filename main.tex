\documentclass[10pt,mathserif]{beamer}

\input defs.tex

%\setbeamerfont*{frametitle}{size=\normalsize,series=\bfseries}

% from boyd
\mode<presentation>
{
\usetheme{default}
}
\setbeamertemplate{navigation symbols}{}
\usecolortheme[rgb={0.13,0.28,0.59}]{structure}
\setbeamertemplate{itemize subitem}{--}
\setbeamertemplate{frametitle} {
	\begin{center}
	  {\large\bf \insertframetitle}
	\end{center}
}

\newcommand\footlineon{
  \setbeamertemplate{footline} {
    \begin{beamercolorbox}[ht=2.5ex,dp=1.125ex,leftskip=.8cm,rightskip=.6cm]{structure}
      \footnotesize \insertsection
      \hfill
      {\insertframenumber}
    \end{beamercolorbox}
    \vskip 0.45cm
  }
}
\footlineon

\newcommand\blfootnote[1]{%
  \begingroup
  \renewcommand\thefootnote{}\footnote{#1}%
  \addtocounter{footnote}{-1}%
  \endgroup
}

\AtBeginSection[] 
{ 
	\begin{frame}<beamer> 
		\frametitle{Outline} 
		\tableofcontents[currentsection,currentsubsection] 
	\end{frame} 
}

%
\usepackage[export]{adjustbox}
\usepackage{centernot}
\usepackage{caption}
\usepackage{subcaption}
\usepackage{booktabs} % for professional tables
\usepackage{multirow}
\usepackage{microtype}
\usepackage{graphicx}
\usepackage{amsmath, amssymb, amsthm}
\usepackage{dsfont}
\usepackage{mathtools}
\usepackage{algorithm,algorithmic}
\usepackage{adjustbox}
% Setup TikZ
\usepackage{tikz}
\usepackage{tkz-graph}
\usepackage{pgfplots}
\usepackage{hyperref}

\usepackage[beamer]{hf-tikz} 
\usetikzlibrary{backgrounds,arrows,shapes.geometric,shapes.misc,positioning,patterns}


\tikzstyle{block}=[draw opacity=0.7,line width=1.4cm]

\renewcommand{\algorithmicrequire}{\textbf{initialize}}
\newcommand{\algorithmicinput}{\textbf{input}}
\newcommand{\algorithmicoutput}{\textbf{output}}
\newcommand{\INPUT}{\item[\algorithmicinput]}
\newcommand{\OUTPUT}{\item[\algorithmicoutput]}

\newtheorem{thm}{Theorem}
\newtheorem{defn}{Definition}

% Author, Title, etc.
\title{PRML Lecture Note2}
\author{Yeonwoo Jeong}
\institute
    {Seoul National University}
\date{2019.03.21}

% hide solutions in handout mode
\newcommand\hideit[1]{%
  \only<0| handout:1>{\mbox{}}%
  \invisible<0| handout:1>{#1}}

% independence symbol
\newcommand{\indep}{\raisebox{0.05em}{\rotatebox[origin=c]{90}{$\models$}}}
\tikzset{above left offset={0.0,0.6},below right offset={0.0,-0.5}}

% The main document
\begin{document}
\begin{frame}
  \titlepage
\end{frame}

\setbeamercolor{block body}{parent=normal text,use=block title,bg=block title.bg!15!bg}
\begin{frame}
\frametitle{Law of total variance}
\begin{align}
\Var_X[X] = \Expect_Y[\Var[X|Y]] + \Var_Y[\Expect_X[X|Y]] \nonumber
\end{align}
\begin{proof}
\begin{align}
\Var_X[X] &= \Expect_X[X^2] - \Expect_X[X]^2 \nonumber\\
&= \Expect_Y\Expect_X[X^2|Y] - \Expect_Y \Expect_{X} [X|Y]^2 \nonumber\\
&= \Expect_Y[\Expect_X[X^2] - \Expect_{X| Y} [X]^2]+\Expect_Y[\Expect_X[X|Y]^2] - \Expect_Y \Expect_X[X|Y]^2 \nonumber\\
&= \Expect_Y \Var_X[X|Y] + \Var_Y\Expect_X[X|Y] \nonumber
\end{align}
\end{proof}
\end{frame}
\begin{frame}
\frametitle{Parameter estimation in parametric distributions}
\begin{itemize}
\item
\begin{align}
\Expect_\theta[\theta] &= \int p(\theta) \theta d\theta \nonumber\\
                   &= \int \int p(\theta | \Dataset) \theta d\theta p(\Dataset) d\Dataset \nonumber\\
                   &= \Expect_{\Dataset}\Expect_\theta[\theta|\Dataset] \nonumber
\end{align}
\item
\begin{align}
\Var_\theta[\theta] = \Expect_\Dataset\Var_\theta[\theta|\Dataset] + \Var_\Dataset\Expect_\theta[\theta|\Dataset]\nonumber
\end{align}
\end{itemize}
\end{frame}
\begin{frame}
\frametitle{Bernoulli distribution}
\begin{itemize}\itemsep=12pt
\item A single binary random variable $x\in \{0, 1\}$
\item $p(x|\mu) = \mu^x(1-\mu)^{1-x}$ for $0\leq \mu \leq 1$
\item $\Expect[x] = \mu$, $\Var[x] = \mu(1-\mu)$
\item For $\Dataset=\{x_i\}_{i=1}^N$, $\mu(=\frac{1}{N}\sum_{i=1}^N x_i)$  maximizes $p(\Dataset|\mu)$(MLE).
\item $\sum_{i=1}^N x_i$ is called a \textit{sufficient statistic}
\end{itemize}
\end{frame}

\begin{frame}
\frametitle{Binomial distribution}
\begin{itemize}\itemsep=12pt
\item Random variable $m=x_1+\ldots+x_N$ for $x_i$s are random variables from iid bernoulli distribution.
\item
\begin{align}
p(m| N,\mu)= {N \choose m} \mu^m (1-\mu)^{N-m}\nonumber
\end{align}
\item
\begin{align}
    \Expect[m]&= N\mu, \Var[m]= N\mu(1-\mu)\nonumber
\end{align}
\end{itemize}
\end{frame}
\begin{frame}
\frametitle{Conjugate prior }
\begin{itemize}\itemsep=12pt
\item Small dataset might result in over-fitted results.
\item Bayesian approach to introduce a prior distribution $p(\mu)$.
\item To make posterior distribution have the same functional form as the prior distribution.
\end{itemize}
\end{frame}
\begin{frame}
\frametitle{Beta distribution}
\begin{itemize}\itemsep=12pt
\item Conjugate prior of Binomial distribution
\begin{align}
p(\mu| a,b) &= \frac{\Gamma(a+b)}{\Gamma(a)\Gamma(b)} \mu^{a-1}(1-\mu)^{b-1}\nonumber\\
\Expect[\mu] &= \frac{a}{a+b} \nonumber\\
\Var[\mu] &= \frac{ab}{(a+b)^2(a+b+1)} \nonumber
\end{align}
\item If $\Dataset$ consists of $m$ `heads' and $l(=N-m)$ `tails', 
\begin{align}
p(\mu | \Dataset,a,b) \propto \mu^{m+a-1}(1-\mu)^{N-m+b-1} \nonumber
\end{align}.
\end{itemize}
\end{frame}
\begin{frame}
\frametitle{A generalization of the Bernoulli Distribution}
\begin{itemize}\itemsep=12pt
\item $\bfmu = (\mu_1,\ldots, \mu_K)^\intercal$
\item Each variable is represented by a $K$-dimensional vector $\bfx$.
\item 
\begin{align}
p(\bfx | \bfmu) = \prod_{k=1}^K \mu_k^{x_k}\nonumber
\end{align}
\item
\begin{align}
\Expect[\bfx | \bfmu] = \sum_\bfx p(\bfx | \bfmu) \bfx = \bfmu\nonumber
\end{align}
\end{itemize}
\end{frame}

\begin{frame}
\frametitle{A generalization of the Bernoulli Distribution}
\begin{itemize}\itemsep=12pt
\item $\Dataset = \{\bfx_1,\ldots,\bfx_N\}$, \textit{sufficient statistic} are 
\begin{align}
m_k= \sum_{n=1}^N \bfx_{n,k}\nonumber
\end{align}
for $k=\{1,\ldots,K\}$.

\item
\begin{align}
    \maximize_{\mu_1,\ldots,\mu_K} p(\Dataset| \bfmu) = \prod_{k=1}^K \mu_k^{m_k}\nonumber
\end{align}
\end{itemize}
\end{frame}
\begin{frame}
\frametitle{Multinomial Distribution}
\begin{itemize}\itemsep=12pt
\item 
\begin{align}
p(m_1,\ldots,m_K|\bfmu, N) = {N \choose m_1,\ldots m_K} \prod_{k=1}^K \bfmu_k^{m_k} \nonumber
\end{align}
\end{itemize}
\end{frame}

\begin{frame}
\frametitle{Dirichlet distribution}
\begin{itemize}\itemsep=12pt
\item Conjugate prior of Multinomial distribution
\item Prior distribution
\begin{align} 
p(\bfmu| \bfalpha) \propto \prod_{k=1}^K \mu_k^{\alpha_k-1} \nonumber
\end{align}
\item Posterior distribution
\begin{align} 
p(\bfmu| \Dataset, \bfalpha) \propto p(\Dataset | \bfmu)p(\bfmu | \alpha) \propto \prod_{k=1}^K \mu_k^{m_k+\alpha_k-1} \nonumber
\end{align}
\end{itemize}
\end{frame}

\begin{frame}
\frametitle{Gaussian distribution}
\begin{itemize}\itemsep=12pt
\item
\begin{align}
\Normal(x | \mu, \sigma^2) = \frac{1}{(2\pi \sigma^2)^{1/2}}\exp\left\{-\frac{1}{2\sigma^2} (x-\mu)^2 \right\}\nonumber
\end{align}
\item 
\begin{align}
\Normal(\bfx | \bfmu, \bfSigma) = \frac{1}{(2\pi)^{D/2}} \frac{1}{|\bfSigma|^{1/2}}\exp\left\{-\frac{1}{2} (\bfx-\bfmu)^\intercal \bfSigma^{-1}(\bfx-\bfmu)\right\}\nonumber
\end{align}
\item Central limit theorem 
\end{itemize}
\end{frame}
\begin{frame}
\frametitle{Characteristic function}
\begin{itemize}\itemsep=12pt
\item $\Phi_X(\omega) = \Expect[e^{j\omega X}]$
\item In case of Gaussian distribution,
\begin{align}
\Phi_X(\omega) &= \Expect[e^{j\omega X}]\nonumber\\
&=\int_{-\infty}^{\infty} \frac{1}{\sqrt{2\pi}\sigma}e^{-\frac{(x-\mu)^2}{2\sigma^2}+j\omega x} dx \nonumber\\
&=e^{j\omega \mu-\omega^2\sigma^2/2}\int_{-\infty}^{\infty} \frac{1}{\sqrt{2\pi}\sigma}e^{-\frac{(x-(\mu+j\omega \sigma^2))^2}{2\sigma^2}} dx =\nonumber\\
&=e^{j\omega \mu-\omega^2\sigma^2/2} \nonumber
\end{align}
\end{itemize}
\end{frame}
      
\begin{frame}
\frametitle{Central limit theorem}
\begin{itemize}\itemsep=12pt
\item $X_k$ : iid random variables from  $\Expect[X_k] = \mu, \Var[X_k]=\sigma^2$.
\item $Z_n$ follows the gaussian distribution mean 0 and variance $\sigma^2$   
\begin{align}
Z_n=\frac{\sum_{k=1}^{n} (X_k-\mu)}{\sqrt{n}}\nonumber
\end{align}
\end{itemize}
\end{frame}
\begin{frame}
    \frametitle{Proof for central limit theorem}
\begin{proof}
\begin{align}
\Phi_{Z_n}(\omega) &= E[e^{j\omega \frac{\sum_{k=1}^{n} (X_k-\mu)}{\sqrt{n}}}]= E[\prod_{k=1}^{n}e^{j\omega \frac{(X_k-\mu)}{\sqrt{n}}}]\nonumber\\ 
&=\prod_{k=1}^{n}E[e^{j\omega \frac{(X_k-\mu)}{\sqrt{n}}}] = (E[e^{j\omega \frac{(X_k-\mu)}{\sqrt{n}}}])^n \nonumber\\
&=(1+j\omega \frac{(X_k-\mu)}{\sqrt{n}}-\omega^2 \frac{(X_k-\mu)^2}{2n}+E[R(\omega)])^n \nonumber
\end{align}
\begin{align}
\lim_{n\to \infty}\Phi_{Z_n}(\omega)&=(1-\frac{\omega^2\sigma^2}{2n})^n= e^{\frac{-\omega^2\sigma^2}{2}} \nonumber \\
\lim_{n\to \infty} P(Z_n = z)&=\frac{1}{\sqrt{2\pi}\sigma}e^{-\frac{z^2}{2\sigma^2}}\nonumber
\end{align}

\end{proof}

\end{frame}

\begin{frame}
\frametitle{Gaussian distribution}
\begin{itemize}\itemsep=12pt
\item $\Delta^2 = (\bfx-\bfmu)^\intercal \bfSigma^{-1} (\bfx - \bfmu)$, \textit{Mahalanobis distance}
\item $\bfSigma$ can be taken to be symmetric.
\item $\bfSigma \bfu_i = \lambda_i \bfu_i, \bfSigma^{-1} \bfu_i=\frac{1}{\lambda_i} \bfu_i$.
\item $\bfu_i^\intercal \bfu_j = \begin{cases} 1 &\text{if $i=j$}\\ 0 &\text{otherwise}\end{cases}$
\item 
\begin{align}
\bfU = \begin{bmatrix} \bfu_1& \cdots& \bfu_d \end{bmatrix}\nonumber
\end{align}
\end{itemize}
\end{frame}

\begin{frame}
\frametitle{Gaussian distribution}
\begin{itemize}\itemsep=12pt
\item
\begin{align}
\Sigma &= \bfU\begin{bmatrix}\lambda_1&& \\ &\ddots&\\&&\lambda_d\end{bmatrix} \bfU^\intercal \nonumber\\
\Sigma^{-1} &= \bfU\begin{bmatrix}\frac{1}{\lambda_1}&&\\&\ddots&\\&&\frac{1}{\lambda_d}\end{bmatrix} \bfU^\intercal \nonumber
\end{align}
\item $\bfy = \bfU^\intercal (\bfx-\bfmu)$
\item 
\begin{align}
\Delta^2 = \bfy^T\begin{bmatrix}\lambda_1&& \\ &\ddots&\\&&\lambda_d\end{bmatrix}\bfy\nonumber
\end{align}
\end{itemize}
\end{frame}

\begin{frame}
\frametitle{Properties for gaussian distribution}
Prove the following properties with the substitution $\bfy = \bfU^\intercal (\bfx-\bfmu)$
\begin{itemize}\itemsep=12pt
\item $\Expect[\bfx] = \bfmu$
\item $\Expect[\bfx\bfx^\intercal] = \mu\mu^\intercal + \bfSigma$
\item $\Cov[\bfx] = \Expect[(\bfx-\Expect[\bfx])(\bfx-\Expect[\bfx])^\intercal] = \bfSigma$
\end{itemize}
\end{frame}

\begin{frame}
\frametitle{Conditional and Marginal distribution}
\begin{itemize}\itemsep=12pt
\item $\bfx=\begin{pmatrix}\bfx_a\\\bfx_b\end{pmatrix}$, $\bfmu = \begin{pmatrix} \bfmu_a\\\bfmu_b\end{pmatrix}, \bfSigma = \begin{pmatrix}\bfSigma_{aa} & \bfSigma_{ab}\\ \bfSigma_{ba} & \bfSigma_{bb} \end{pmatrix}$
\item $\bfLambda = \bfSigma^{-1}=\begin{pmatrix}\bfLambda_{aa} & \bfLambda_{ab} \\ \bfLambda_{ba} & \bfLambda_{bb}\end{pmatrix}$
\item $p(\bfx_a| \bfx_b) = \Normal\left(\bfx_a| \bfmu_a - \bfLambda_{aa}^{-1}\bfLambda_{ab}(\bfx_b-\bfmu_b), \bfLambda_{aa}^{-1}\right)$ 
\item $p(\bfx_a) = \Normal\left(\bfx_a| \bfmu_a, \bfSigma_{aa} \right)$ 
\end{itemize}
\end{frame}
\begin{frame}
\frametitle{Bayes' theorem for Gaussian variables}
\begin{itemize}\itemsep=12pt
\item When 
    \begin{align}
        p(\bfx) &= \Normal(\bfx | \bfmu, \bfLambda^{-1}) \nonumber\\
        p(\bfy| \bfx) &= \Normal(\bfy | \bfA\bfx+\bfb, \bfL^{-1}) \nonumber
    \end{align},
\item Prove
\begin{align}
&p(\bfy) = \Normal(\bfy| \bfA\bfx + \bfb, \bfL^{-1}+\bfA\bfLambda^{-1}\bfA^{\intercal})\nonumber\\
&p(\bfx| \bfy)\nonumber\\
&= \Normal(\bfx| (\bfLambda+\bfA^{\intercal}\bfL\bfA )^{-1}\{\bfA^{\intercal}\bfL(\bfy-\bfb)+\bfLambda\bfmu \}, (\bfLambda+\bfA^{\intercal}\bfL\bfA )^{-1})\nonumber
\end{align}
\end{itemize}
\end{frame}
\begin{frame}
\frametitle{MLE on Gaussian distribution}
\begin{itemize}\itemsep=12pt
    \item 
        \begin{align}
            &\maximize_{\bfmu, \bfSigma} \ln p(\bfX|\bfmu, \bfSigma)\nonumber\\
            &=-\frac{ND}{2} \ln 2\pi - \frac{N}{2} \ln |\bfSigma| - \frac{1}{2} \sum_{n=1}^N (\bfx_n-\bfmu)^\intercal \bfSigma^{-1} (\bfx_n-\bfmu) \nonumber
        \end{align}
    \item Prove
        \begin{align}
            \bfmu_{ML} &= \frac{1}{N} \sum_{n=1}^N \bfx_n\nonumber\\
            \bfSigma_{ML} &= \frac{1}{N} \sum_{n=1}^N (\bfx_n - \bfmu_{ML})(\bfx_n - \bfmu_{ML})^\intercal\nonumber
        \end{align}
\end{itemize}
\end{frame}
\begin{frame}
\frametitle{Tips on matrix derivative}
\begin{itemize}\itemsep=12pt
    \item 
        \begin{align}
            \frac{d|\bfA|}{d\bfA} = |\bfA|\bfA^{-1}\nonumber
        \end{align}
    \item 
        \begin{align}
            \frac{d\left(\bfx^\intercal \bfA \bfx \right)}{d\bfA} = \bfx\bfx^\intercal \nonumber
        \end{align}

\end{itemize}
\end{frame}
\begin{frame}
\frametitle{Robbins-Monro algorithm}
\begin{itemize}\itemsep=12pt
    \item $f(\theta) = \Expect [z|\theta] = \int z p(z|\theta) dz$ and we can sample $z$ when $\theta$ is given.
    \item Our goal is to find $\theta^{*}$ such that $f(\theta^{*}) = 0$.
    \item Additional condition : $\Var[z|\theta]$ is finite.
\end{itemize}
\end{frame}

\begin{frame}
\frametitle{Robbins-Monro algorithm}
\begin{itemize}\itemsep=12pt
    \item $\theta^{(N)} = \theta^{(N-1)}+a_{N-1}z_N(\theta^{(N-1)})$
    \item $z_N(\theta^{(N-1)})$ is a sampling from $p(z|\theta^{(N-1)})$.
    \item $\{a_N\}$ should satisfy the conditions. e.g. $a_N=\frac{1}{N}$
    \begin{enumerate}\itemsep=12pt
        \item $\lim_{N\rightarrow \infty} a_N = 0$ :  process converges to a limiting value.
        \item $\sum_{N=1}^{\infty} a_N = \infty$ : does not converge short of the root
        \item $\sum_{N=1}^{\infty} a_N^2 < \infty$ : accumulated noise should have a finite noise.
    \end{enumerate}
\end{itemize}
\end{frame}

\begin{frame}
\frametitle{Application of Robbins-Monro algorithm : sequential estimation}
\begin{itemize}\itemsep=12pt
\item Objective of MLE is 
\begin{align}
\maximize_{\theta} \frac{1}{N} \sum_{n=1}^N \ln p(x_n|\theta), \nonumber
\end{align}
when $\Dataset = \{x_n\}_{n=1}^N$.
\item Our goal is find $\theta_{ML}$ such that
\begin{align}
    \Expect_x \left[\frac{\partial}{\partial \theta} \ln p(x|\theta)\right] = \lim_{N\rightarrow \infty}\frac{\partial}{\partial \theta} \left\{\frac{1}{N} \sum_{n=1}^N \ln p(x_n|\theta) \right\} = 0\nonumber
\end{align}
\item $z = g(x,\theta) = \frac{\partial}{\partial \theta} \ln p(x|\theta)$
\end{itemize}
\end{frame}
\begin{frame}
\frametitle{Application of Robbins-Monro algorithm : sequential estimation}
\begin{itemize}\itemsep=12pt
\item $f(\theta) = \Expect [z|\theta] = 0$
\item $z = g(x,\theta) = \frac{\partial}{\partial \theta} \ln p(x|\theta)$
\end{itemize}
\end{frame}

\begin{frame}
\frametitle{Sequential estimation}
\begin{itemize}\itemsep=12pt
\item E.g. sequential estimation of $\mu_{ML}$
\begin{align}
    z(x,\mu) = \frac{\partial}{\partial \mu}\ln p(x|\mu,\sigma^2) = \frac{1}{\sigma^2}(x-\mu)\nonumber
\end{align}
\item 
\begin{align}
    z_N|\mu^{(N-1)} = \frac{1}{\sigma^2} (x_N- \mu^{(N-1)})\nonumber
\end{align}
\item 
\begin{align}
\mu^{(N)} &= \mu^{(N-1)} + a_Nz_N|\mu^{(N-1)}\nonumber\\
&= \mu^{(N-1)} + a_N\frac{1}{\sigma^2} (x_N- \mu^{(N-1)})\nonumber
\end{align}

\end{itemize}
\end{frame}

\end{document}


