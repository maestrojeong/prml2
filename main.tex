\documentclass[10pt,mathserif]{beamer}

\input defs.tex

%\setbeamerfont*{frametitle}{size=\normalsize,series=\bfseries}

% from boyd
\mode<presentation>
{
\usetheme{default}
}
\setbeamertemplate{navigation symbols}{}
\usecolortheme[rgb={0.13,0.28,0.59}]{structure}
\setbeamertemplate{itemize subitem}{--}
\setbeamertemplate{frametitle} {
	\begin{center}
	  {\large\bf \insertframetitle}
	\end{center}
}

\newcommand\footlineon{
  \setbeamertemplate{footline} {
    \begin{beamercolorbox}[ht=2.5ex,dp=1.125ex,leftskip=.8cm,rightskip=.6cm]{structure}
      \footnotesize \insertsection
      \hfill
      {\insertframenumber}
    \end{beamercolorbox}
    \vskip 0.45cm
  }
}
\footlineon

\newcommand\blfootnote[1]{%
  \begingroup
  \renewcommand\thefootnote{}\footnote{#1}%
  \addtocounter{footnote}{-1}%
  \endgroup
}

\AtBeginSection[] 
{ 
	\begin{frame}<beamer> 
		\frametitle{Outline} 
		\tableofcontents[currentsection,currentsubsection] 
	\end{frame} 
}

%
\usepackage[export]{adjustbox}
\usepackage{centernot}
\usepackage{caption}
\usepackage{subcaption}
\usepackage{booktabs} % for professional tables
\usepackage{multirow}
\usepackage{microtype}
\usepackage{graphicx}
\usepackage{amsmath, amssymb, amsthm}
\usepackage{dsfont}
\usepackage{mathtools}
\usepackage{algorithm,algorithmic}
\usepackage{adjustbox}
% Setup TikZ
\usepackage{tikz}
\usepackage{tkz-graph}
\usepackage{pgfplots}
\usepackage{hyperref}

\usepackage[beamer]{hf-tikz} 
\usetikzlibrary{backgrounds,arrows,shapes.geometric,shapes.misc,positioning,patterns}


\tikzstyle{block}=[draw opacity=0.7,line width=1.4cm]

\renewcommand{\algorithmicrequire}{\textbf{initialize}}
\newcommand{\algorithmicinput}{\textbf{input}}
\newcommand{\algorithmicoutput}{\textbf{output}}
\newcommand{\INPUT}{\item[\algorithmicinput]}
\newcommand{\OUTPUT}{\item[\algorithmicoutput]}

\newtheorem{thm}{Theorem}
\newtheorem{defn}{Definition}

% Author, Title, etc.
\title{PRML Lecture Note2}
\author{Yeonwoo Jeong}
\institute
    {Seoul National University}
\date{2017.09.01}

% hide solutions in handout mode
\newcommand\hideit[1]{%
  \only<0| handout:1>{\mbox{}}%
  \invisible<0| handout:1>{#1}}

% independence symbol
\newcommand{\indep}{\raisebox{0.05em}{\rotatebox[origin=c]{90}{$\models$}}}
\tikzset{above left offset={0.0,0.6},below right offset={0.0,-0.5}}

% The main document
\begin{document}
\begin{frame}
  \titlepage
\end{frame}

\setbeamercolor{block body}{parent=normal text,use=block title,bg=block title.bg!15!bg}
\begin{frame}
    \frametitle{Law of total variance}
    \begin{align}
        \Var_x[x] = \Expect_y[\Var[x\mid y]] + \Var_y[\Expect_x[x\mid y]] \nonumber
    \end{align}
    \begin{proof}
        \begin{align}
            \Var_x[x] &= \Expect_x[x^2] - \Expect_x[x]^2 \nonumber\\
            &= \Expect_y\Expect_{x\mid y}[x^2] - \Expect_y \Expect_{x\mid y} [x]^2 \nonumber\\
            &= \Expect_y[\Expect_{x\mid y}[x^2] - \Expect_{x\mid y} [x]^2]+\Expect_y[\Expect_{x\mid y}[x]^2] - \Expect_y \Expect_{x|y}[x]^2 \nonumber\\
            &= \Expect_y[\Expect_{x\mid y}[x^2] - \Expect_{x\mid y} [x]^2]+\Expect_y[\Expect_{x\mid y}[x]^2] - \Expect_y \Expect_{x|y}[x]^2 \nonumber\\
            &= \Expect_y \Var_{x\mid y}[x] + \Expect_y[\Expect_{x\mid y}[x]^2] - \Expect_y \Expect_{x|y}[x]^2 \nonumber
        \end{align}
    \end{proof}
\end{frame}
%%\begin{frame}
%%    \frametitle{Form1\footnote{{\color{blue}{blue}} ``Footnote example``. \textcolor{gray}{gray.}}}
%%    \begin{figure}
%%        \centering
%%        \includegraphics[width=0.9\textwidth,center]{npair}
%%    \end{figure}
%%    \vspace{1em}
%%    \begin{itemize}\itemsep=12pt
%%        \item Vspace is to control space between figure and items. 
%%        \item Itemsep to control space between lines.
%%    \end{itemize}
%%\end{frame}
%%
%%\begin{frame}
%%    \frametitle{Form2}
%%    \begin{align}
%%        &\minimize_{\substack{\bfh_1, \ldots, \bfh_n\\\bfh_i \in \{0,1\}^d, || \bfh_i ||_1 = k}} ~\sum_{i}^{n_c}\sum_{k:y_k=i}-\bfc_i^\intercal \bfh_k+\sum_{i}^{n_c} \sum_{\substack{k:y_k=i,\\l:y_l \neq i}} \bfh_k^\intercal P \bfh_l \\
%%    \end{align}
%%    \begin{itemize}
%%        \item Equation example
%%    \end{itemize}
%%\end{frame}
%%
%%\begin{frame}
%%    \frametitle{Form3} 
%%    \begin{columns}
%%        \begin{column}{0.5\textwidth}
%%            \begin{figure}
%%                \includegraphics[width=\textwidth,center]{npair}
%%            \end{figure}
%%        \end{column}
%%        \begin{column}{0.5\textwidth}
%%            \begin{figure}[ht]
%%                \includegraphics[width=0.5\textwidth,center]{npair}
%%            \end{figure}
%%        \end{column}
%%    \end{columns}
%%    \begin{itemize}\itemsep=12pt
%%        \item Form by multiicolumn
%%    \end{itemize}
%%    \normalsize
%%\end{frame}
\end{document}


